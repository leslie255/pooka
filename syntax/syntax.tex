\documentclass[a4paper]{article}
\title{Formal (Somewhat) Definition for Pooka Language Syntax}
\author{Leslie}
\usepackage[
    a4paper,
    left=2cm,
    right=2cm,
    top=2cm,
    bottom=2cm,
    footskip=0.1cm]{geometry}
\usepackage{amsmath}
\usepackage{gensymb}

\begin{document}

\maketitle

\section{Lexical Elements}

\subsection{Identifiers}

\begin{verbatim}
    <ident>
\end{verbatim}

An identifier is a string of characters that starts with one of the below:

\begin{itemize}
    \item A UTF-8 alphanumeric that is not an ASCII numeric
    \item A UTF-8 emoji
    \item An ASCII underscore
\end{itemize}

An identifier can then be followed by one of the below:

\begin{itemize}
    \item Any of the above
    \item An ASCII numeric
    \item An ASCII single quote
\end{itemize}

An identifier cannot be a boolean literal or a keyword.

\subsection{Literals}

\begin{verbatim}
    <lit> := <int lit>|<float lit>|<string lit>|<char lit>|<bool lit>
\end{verbatim}

\subsubsection{Integer Literals}

\begin{verbatim}
    <int lit>
\end{verbatim}

An integer literal starts with either a base prefix or an ASCII numeric, as listed below:

\begin{enumerate}
    \item \verb|0b|: Binary
        \label{base_prefix_0b}
    \item \verb|0o|: Octal
    \item \verb|0d|: Decimal
    \item \verb|0x|: Hexadecimal
        \label{base_prefix_0x}
    \item Any ASCII numeric: Decimal
        \label{no_base_prefix}
\end{enumerate}

\begin{itemize}
    \item For case \ref{base_prefix_0b} to \ref{base_prefix_0x}:
    \begin{itemize}
        \item It must be followed by a string of characters of that specific base.
    \end{itemize}
    \item For case \ref{no_base_prefix}:
    \begin{itemize}
        \item It may then be followed by a string of ASCII numerics.
        \item It may then be followed by the character \verb|e|, if so, it must then be followed by a string of ASCII numerics.
    \end{itemize}
\end{itemize}

If a sequence of character start with an ASCII numeric, and it otherwise satisfies the requirement of being an identifier, it is an invalid integer literal.

\subsubsection{Floating Point Literals}

\begin{verbatim}
    <float lit>
\end{verbatim}

A floating point literal starts with an ASCII numeric, followed by an ASCII period, followed by another string of ASCII numerics.

It may then be followed by the character \verb|e|, if so, it must then be followed by a string of ASCII numerics.

\subsubsection{Character Literals}

\begin{verbatim}
    <char lit>
\end{verbatim}

A character literal starts with an ASCII single quote sign, followed by either one UTF-8 character or an string escape sequence, followed by another single quote sign.

\subsubsection{String Literals}

\begin{verbatim}
    <string lit>
\end{verbatim}

A string literal starts with an ASCII double quote sign, followed by a sequence of either UTF-8 characters or string escape sequence, followed by another single quote sign.

\subsubsection{String Escape Sequence}

Valid string escape sequences are:

\begin{itemize}
    \item \verb|\n|: Newline
    \item \verb|\r|: Carrige return
    \item \verb|\t|: Horizontal tab
    \item \verb|\v|: Vertical tab
    \item \verb|\\|: Backslash
    \item \verb|\0|: Null
    \item \verb|\x|, followed by 2 hexadecimal digits
    \item \verb|\u{|, followed by 2 to 6 hexadecimal digits, followed by \verb|}|
\end{itemize}

\subsubsection{Boolean Literal}

\begin{verbatim}
    <bool lit>
\end{verbatim}

Boolean literals are \verb|true| or \verb|false|.

\subsection{Macro Directives}

\begin{verbatim}
    <@ident>
\end{verbatim}

Macro directive starts with a \verb|@|, it can then be followed by:

\begin{itemize}
    \item A UTF-8 alphanumeric that is not an ASCII numeric
    \item A UTF-8 emoji
    \item An ASCII underscore
    \item An ASCII single quote
\end{itemize}

\subsection{Parenthesis}

A parenthesis is one of the following characters:

\begin{itemize}
    \item \verb|{|
    \item \verb|}|
    \item \verb|[|
    \item \verb|]|
    \item \verb|(|
    \item \verb|)|
\end{itemize}

\subsection{Punctuations}

\begin{verbatim}
    <punct> := <unreserved punct> | <reserved punct>
\end{verbatim}

A punctuation is a sequence of characters that satisify {\bf all} follow requirements:

\begin{itemize}
    \item Is a UTF-8 punctuation character
    \item Is not a \verb|<paren>|
    \item Is not an ASCII single quote or double quote
\end{itemize}

Some punctuations are reserved (\verb|<reserved punct>|), the rest are not (\verb|<unreserved punct>|).

This distinction is made to make room in the syntax for a potential future expansion of custom operators, for this reason occurance of \verb|<unreserved punct>| in source code is considered invalid in the current version.

\begin{itemize}
    \item \verb|,|
    \item \verb|.|
    \item \verb|=|
    \item \verb|:=|
    \item \verb|:|
    \item \verb|::|
    \item \verb|*|
    \item \verb|~|
    \item \verb|&|
    \item \verb/|/
    \item \verb|^|
    \item \verb|>>|
    \item \verb|<<|
    \item \verb|>>=|
    \item \verb|<<=|
    \item \verb|&=|
    \item \verb/|=/
    \item \verb|^=|
    \item \verb|!|
    \item \verb|&&|
    \item \verb/||/
    \item \verb|+|
    \item \verb|-|
    \item \verb|/|
    \item \verb|%|
    \item \verb|+=|
    \item \verb|-=|
    \item \verb|*=|
    \item \verb|/=|
    \item \verb|%=|
    \item \verb|>|
    \item \verb|<|
    \item \verb|>=|
    \item \verb|<=|
    \item \verb|==|
    \item \verb|!=|
    \item \verb|@|
\end{itemize}

\pagebreak

\section{Syntax}

\subsection{Types}

\begin{verbatim}
    <ty> := <ident>
          | &<ident>
          | &mut <ident>
          | [<ty>]
          | &mut [<ty>]
          | ({<ty>},*)
          | struct \{ <pat ty pairs> \}
          | union \{ <pat ty pairs> \}
          | enum \{ {<ident>(<ty>)},* \}
\end{verbatim}

\subsection{Patterns}

\begin{verbatim}
    <pat> := <ident>
           | mut <ident>
           | ({<pat>},*)
\end{verbatim}

\begin{verbatim}
    <pat ty pairs> := {<pat>:<ty>},*
\end{verbatim}

\subsection{Expressions}

\begin{verbatim}
    <expr> := <ident>
            | <lit>
            | <op expr>
            | ({<expr>},*)
\end{verbatim}

\subsection{Assignments}

\begin{verbatim}
    <assign> := <pat> = <expr> ;
\end{verbatim}

\begin{verbatim}
    <expr> := <ident>
            | <lit>
            | <op expr>
            | ({<expr>},*)
            | <if>
            | <loop>
\end{verbatim}

\subsection{Statements and Blocks}

\begin{verbatim}
    <stmt> := <expr> ;
            | <var decl>
            | <fn decl>
            | <typealias>
            | <newtype>
            | <if>
            | <loop>
            | <while>
\end{verbatim}

\begin{verbatim}
    <block> := \{ {<stmt>}* \};
\end{verbatim}

\subsection{Variable Declarations}

\begin{verbatim}
    <var decl> := <pat> : <ty> {= <expr>}? ;
                | <pat> {:=}|{: =} <expr> ;
\end{verbatim}

\subsection{Function Declarations}

\begin{verbatim}
    <fn decl> := <ident> :: (<pat ty pairs>) {= <expr> ;}|{<block>}|{;}
\end{verbatim}

\subsection{Type/Typealias Statements}

\begin{verbatim}
    <newtype> := type <ident> = <ty> ;
\end{verbatim}

\begin{verbatim}
    <typealias> := typealias <ident> = <ty> ;
\end{verbatim}

\subsection{If}

\begin{verbatim}
    <if> := if <expr> <block> {else <block>}?
\end{verbatim}

\subsection{Loop}

\begin{verbatim}
    <loop> := loop <block>
\end{verbatim}

\subsection{While}

\begin{verbatim}
    <while> := while <expr> <block>
\end{verbatim}

\end{document}

